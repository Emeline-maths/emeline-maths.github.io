\documentclass[12pt,a4paper]{article}
\usepackage[utf8x]{inputenc}
\usepackage{ucs}
\usepackage[francais]{babel}
\usepackage[T1]{fontenc}
\usepackage{amsmath}
\usepackage{amsfonts}
\usepackage{amsthm}
\usepackage{hyperref}
\usepackage{textcomp}
\usepackage{xcolor}
\theoremstyle {definition} \newtheorem {defi} {Définition} [section]
\newtheorem{ex}{Exemple}[section]
\theoremstyle{plain} \newtheorem {theo} {Théorème}[section]
\newtheorem {coro}[theo]{Corollaire}
\newtheorem{prop}{Proposition}[section]
\newtheorem {lem} {Lemme}[section]
\theoremstyle{remark} \newtheorem*{note}{Remarque}
\DeclareMathOperator{\GL}{GL}
\DeclareMathOperator{\SL}{SL}
\usepackage{graphicx}
\usepackage{amssymb}
\usepackage{pgf, tikz} %pour les graphes
\usepackage[margin=15mm]{geometry}
\usetikzlibrary{3d,calc}
\usetikzlibrary{trees}
%\usepackage[left=2cm,right=2cm,top=2cm,bottom=2cm]{geometry}
%\usepackage{float}
%\usepackage{ulem}
%\usepackage[all]{xy}
\usepackage{hyperref}
\usepackage{upquote}
\newcommand{\egale}[1]{%
\ensuremath{\stackrel{#1}{=}}} % pour ecrire au dessus du égal
\newcommand{\abs}[1]{\left\lvert#1\right\rvert}
\newcommand{\ii}{\text{i}}

\title{Site internet}
\author{Emeline LUIRARD}

\begin{document}

\maketitle	
Pour accéder via une connexion internet à son espace personnel, on utilise ssh avec le shell:
\begin{verbatim}
ssh <login>@ssh.eleves.ens-rennes.fr
\end{verbatim}

où \verb|<login>| est l'identifiant ENS: \verb|eluir933| par exemple.
Il est ensuite demandé un mot de passe: c'est celui qu'on utilise pour se connecter à l'ENS. 
Ensuite dans le terminal on a \verb|eluir933@darwin:~\$ | ce qui veut dire qu'on est sur notre dossier.

Pour copier un document: se placer dans le répertoire courant où est ce document
\begin{verbatim}
	scp document <login>@ssh.eleves.ens-rennes.fr:
\end{verbatim} permet de copier \verb|document| vers son espace perso. 
On peut préciser un emplacement de destination comme cela:
\begin{verbatim}
scp exercice.pdf eluir933@ssh.eleves.ens-rennes.fr:public_html/Fichiers_pdf/
\end{verbatim}
Pour copier tout un dossier, il faut utiliser l'option \verb|-r|:
\begin{verbatim}
	scp -r Documents_pdf/ eluir933@ssh.eleves.ens-rennes.fr:public_html/
\end{verbatim}

Maintenant place aux commandes du shell pour se débrouiller:

\verb|ls|: permet de lister les fichiers et les sous-dossiers.
\\ 
\verb|cd public_html| permet d'aller dans le dossier \verb|public_html| qu'on voit dans en ayant fait \verb|ls|.
\\ Pour remonter d'un répertoire: \verb|cd ..|  attention à ne pas oublier l'espace.
\\
Pour copier: \verb|cp|
Pour supprimer : \verb|rm|  (attention c'est irrémédiable !)
\begin{verbatim}
rm -i Agregation.html 
\end{verbatim}
pour supprimer le fichier \verb|Agregation.html| dans le dossier courant. Il demande confirmation avec \verb|-i|.
Le reste ici :\url{https://doc.ubuntu-fr.org/tutoriel/console_ligne_de_commande}

 Pour récupérer le dossier \verb|public_html| dans le répertoire courant du terminal(de notre ordinateur à nous), écrire dans un nouveau terminal :
 \begin{verbatim}
 scp -r eluir933@ssh.eleves.ens-rennes.fr:~/public\_html /home/emeline/Documents
 \end{verbatim}
 Pour créer un dossier dans le répertoire courant 
 \begin{verbatim}
 mkdir nouveaudossier
 \end{verbatim}
%$scp -r eluir933@ssh.eleves.ens-rennes.fr:~/public\_html/ ~/$

Il est demandé le mdp de l'ordinateur distant.
\end{document}